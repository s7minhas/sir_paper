\section*{\textbf{Conclusion}}

In this paper, we introduced the Social Influence Regression (SIR) model, which represents an important extension of the bilinear network autoregression framework, designed to more effectively capture and explain influence dynamics within networks. The SIR model addresses a key limitation of existing models by incorporating exogenous covariates into the estimation process, allowing us to directly model and interpret the factors driving influence within a network. This approach not only enhances the explanatory power of the model but also provides a more rigorous and theoretically grounded framework for understanding complex relational data. A key contribution of our work is the development of a more efficient estimation scheme for the SIR model. Using an iterative block coordinate descent method, we enhance the model's computational feasibility, especially for large-scale networks. 

The application of the SIR model to the study of material conflict between countries provided several important insights that underscore the model's practical utility. By incorporating covariates such as geographic proximity, alliances, trade, and verbal cooperation, the SIR model revealed nuanced patterns of influence within the international conflict network. For example, the model identified that countries tend to initiate conflicts against the same targets as their allies, a finding that aligns with established theories in international relations about the role of alliances in escalating conflicts. Additionally, the negative influence of trade flows on conflict initiation suggested that countries are less likely to follow their trading partners into conflict, highlighting the stabilizing effect of economic interdependence. Verbal cooperation was shown to have a reinforcing effect, where countries that publicly support each other are more likely to align their conflictual actions. These findings not only validate the robustness of the SIR model but also demonstrate its ability to generate new theoretical and empirical insights into the dynamics of international conflict, providing a clearer understanding of the factors driving influence within complex networks.

Looking forward, the SIR model opens up numerous avenues for future research, both in terms of its applications and methodological developments. The model's flexibility allows it to be adapted to various network contexts beyond international conflict. Methodologically, there are several promising directions for refinement and extension. One area for development is the further optimization of the block coordinate descent method, particularly for handling even larger and more complex networks. This could involve parallelizing the estimation process or incorporating advanced optimization techniques such as stochastic gradient descent to improve scalability and convergence speed.
