\section*{\textbf{Motivation}}

Network influence shapes political outcomes across diverse domains, yet precisely measuring and explaining these patterns of influence remains a challenge. We define network influence broadly as the impact that one actor's actions or decisions have on the behavior of others within a network, whether through direct or indirect connections. Numerous studies have demonstrated the critical role of network influences in explaining a wide range of political phenomena, from subnational policy diffusion to interstate conflict dynamics \citep{cranmer:etal:2015a, beardsley:etal:2020, nieman:etal:2021, edgerton:2024}. A prominent approach to measuring influence relies on latent variable models, which position actors in a social space based on mechanisms such as transitivity and/or stochastic equivalence \citep{gade:etal:2019, huhe:etal:2021, edgerton:2024}. However, while these models can effectively describe the overall structure of a network, they frequently fall short in providing detailed explanations for the specific influence that actors exert on one another. This limitation arises because these models typically attribute influence to broad network features, without accounting for the exogenous factors that might drive such influence, leaving the mechanisms behind these interactions underexplored and poorly understood.

To address this limitation, we propose a novel approach: the Social Influence Regression (SIR) model. This model extends vector autoregression techniques to relational data, allowing us to capture the influence of one actor on another over time while simultaneously incorporating exogenous covariates. The SIR model operates by estimating a pair of $n \times n$ matrices that measure sender- and receiver-level influence patterns, taking the form:

\[
y_{i,j,t} 
\;=\; \sum_{i'=1}^{n} \sum_{j'=1}^{n} a_{i,i'} \, b_{j,j'} \, x_{i',j',t-1}
\;+\; e_{i,j,t},
\]

where $a_{i,i'}$ describes how actor $i'$'s past actions (at time $t-1$) help predict actor $i$'s actions at the current time $t$.\footnote{Here, $(i,j)$ identifies the dyad of interest at time $t$, while $(i',j')$ indexes every dyad from time $t-1$. This means $(i,j)$ itself is part of the summation, which allows $y_{i,j,t}$ to depend on its own lag $y_{i,j,t-1}$.} Similarly, $b_{j,j'}$ indicates how prior actions directed at $j'$ shape the actions now directed at $j$. The term $x_{i',j',t-1}$ represents lagged predictors of the interaction, and $e_{i,j,t}$ is an error term.

The key innovation of the SIR model is its ability to explicitly account for the role of exogenous covariates in shaping these influence patterns. While traditional models in this vein are effective at uncovering patterns within the network, they often fall short of explaining these patterns in terms of observable actor-level or dyad-level attributes. The SIR model bridges this gap by linking actor positioning in the latent social space directly to exogenous covariates, thus providing a more interpretable and theoretically grounded understanding of network dynamics.

Similarly, popular network approaches such as ERGMs and SAOMs, while adept at capturing structural dependencies (e.g., transitivity, reciprocity), often leave open questions about the exogenous drivers underlying evolving patterns of influence. In most ERGM specifications, for example, explanatory variables enter as global effects that apply uniformly across all nodes, rather than revealing how these factors might differentially shape specific actors' relational patterns. In contrast, SIR specifically centers on identifying how measured actor- or dyad-level attributes explain who influences whom, and to what extent. By embedding bilinear terms in a regression framework, it offers a means of mapping exogenous covariates --- such as alliances or proximity --- onto dynamic, node-specific influence processes, enabling researchers to investigate why certain actors wield disproportionate control or how conflict may cluster among particular sets of states. SIR thus focuses on exogenous triggers that guide or constrain influence, providing answers to a different set of questions about how external conditions catalyze or moderate these relationships over time. This emphasis on covariate-driven explanation broadens the analytic toolkit for network research and offers a direct avenue to see how observable traits guide influence within the network.

We apply this approach to data from the Integrated Crisis Early Warning System (ICEWS) event data project. Using the SIR model, we estimate the extent to which actors within the material conflict network influence one another and, crucially, explore how characteristics such as alliances or economic relationships explain the observed influence patterns. Our findings demonstrate that the SIR model significantly improves out-of-sample performance compared to existing methods. This improvement underscores the model's effectiveness and offers new insights into the drivers of influence in international relations. By providing a more precise and interpretable representation of network dynamics, this work advances both the methodology of network analysis and the substantive understanding of international conflict processes.

The rest of the paper proceeds as follows. In Section 2, we introduce the model in detail, describing its theoretical foundations and estimation procedure. Section 3 presents our empirical application to the ICEWS data, including a description of the data, model specification, and results. We pay particular attention to how alliance relationships and trade flows influence conflict behavior. Section 4 provides a performance comparison of the SIR model against alternative approaches, demonstrating its superior out-of-sample predictive power. Finally, Section 5 concludes with a discussion of the implications of our findings for international relations theory and suggestions for future research directions in network analysis within political science.
